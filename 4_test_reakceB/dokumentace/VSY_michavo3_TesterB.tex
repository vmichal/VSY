\documentclass[twoside]{article}
\usepackage[a4paper]{geometry}
\geometry{verbose,tmargin=2.5cm,bmargin=2cm,lmargin=2cm,rmargin=2cm}
\usepackage{fancyhdr}
\pagestyle{fancy}

% nastavení pisma a~češtiny
\usepackage{lmodern}
\usepackage[T1]{fontenc}
\usepackage[utf8]{inputenc}
\usepackage[czech]{babel}

% odkazy
\usepackage{url}

\usepackage{float}
% vícesloupcové tabulky
\usepackage{multirow}
\usepackage{listings}
\usepackage{xcolor}
\usepackage{amssymb}
\usepackage{gensymb}
\usepackage{bbold}
\usepackage{amsmath}
\usepackage{siunitx}
\usepackage{mathtools}
\usepackage{commath}

% vnořené popisky obrázků
\usepackage{subcaption}

% automatická konverze EPS 
\usepackage{graphicx} 
\usepackage{epstopdf}
\epstopdfsetup{update}

\graphicspath{{./images}}

% odkazy a~záložky
\usepackage[unicode=true, bookmarks=true,bookmarksnumbered=true,
bookmarksopen=false, breaklinks=false,pdfborder={0 0 0},
pdfpagemode=UseNone,backref=false,colorlinks=true] {hyperref}


% Poznámky při překladu
\usepackage{xkeyval}	% Inline todonotes
\usepackage[textsize = footnotesize]{todonotes}
\presetkeys{todonotes}{inline}{}

%https://tex.stackexchange.com/questions/2783/bold-calligraphic-typeface
\DeclareMathAlphabet\mathbfcal{OMS}{cmsy}{b}{n}

% enumerate zacina s pismenem
\renewcommand{\theenumi}{\alph{enumi}}

% smaz aktualni page layout
\fancyhf{}
% zahlavi
\usepackage{titling}
\fancyhf[HC]{\thetitle}
\fancyhf[HLE,HRO]{\theauthor}
\fancyhf[HRE,HLO]{\today}
 %zapati
\fancyhf[FLE,FRO]{\thepage}

% údaje o autorovi
\title{VSY - Pokročilý tester reakce (B) - dokumentace}
\author{Vojtěch Michal}
\date{\today}

%customize code listing
\definecolor{codegreen}{rgb}{0,0.6,0}
\definecolor{codegray}{rgb}{0.5,0.5,0.5}
\definecolor{codepurple}{rgb}{0.58,0,0.82}
\definecolor{backcolour}{rgb}{0.95,0.95,0.92}

\lstdefinestyle{mystyle}{
    backgroundcolor=\color{backcolour},   
    commentstyle=\color{codegreen},
    keywordstyle=\color{magenta},
    numberstyle=\tiny\color{codegray},
    stringstyle=\color{codepurple},
    basicstyle=\ttfamily\footnotesize,
    breakatwhitespace=false,         
    breaklines=true,                 
    captionpos=b,                    
    keepspaces=true,                 
    numbers=left,                    
    numbersep=5pt,                  
    showspaces=false,                
    showstringspaces=false,
    showtabs=false,                  
    tabsize=2
}

\lstset{style=mystyle}

\begin{document}

\maketitle

Cílem úlohy je realizovat tester rychlosti reakce uživatele. Mačkáním tlačítek uživatel reaguje na rozsvěcení indikátorových LED.
Po testu dostane uživatel komplexní statistiku o svém výkonu - průměrnou rychlost reakce, počet správných stisků atd.

\section{Chování}
\label{sec:chovani}

Stav aplikace je indikován rozvěcením připojených LEDek. Po inicializaci čeká aplikace
na začátek testu a obě diody blikají v protifázi. Test lze zahájit buďto přes terminálové
rozhraní, nebo současným stiskem obou tlačítek. Během testu se náhodně rozsvěcí jedna z diod,
na což má uživatel co nejrychleji reagovat stiskem příslušného tlačítka. Pakliže je stisk úspěsný,
LED zhasne a čeká se na další rozvícení. Pakliže uživatel stiskne nesprávné tlačítko, obě diody 
jsou na okamžik rozsvíceny, než se pokračuje v testu. Test lze ukončit souběžným stiskem obou
ovládacích tlačítek, nebo z terminálu. Průběh testu je možno přes terminál kdykoli pozastavit.

\section{Komunikační rozhraní a ovládání}

Aplikace komunikuje obousměrně po seriové lince UART s nastavením baudrate 115200,
osm datových bitů, jeden stop bit, bez parity. Použitá periferie je UART2 (piny PA2, PA3),
ST-LINK převádí komunikaci na USB.

Terminálová aplikace přijímá vstup kdykoli. Mapování znaků na funkce aplikace je v tabulce \ref{table:commands}.
Příkazy nejsou citlivé na velikost písmen, malá i velká písmena fungují ekvivalentně.

\begin{table}[htbp]
    \centering
    \begin{tabular}{c|c|c}
        znak(y) & funkce \\
        q & Ukončení probíhající serie testu \\
        r & Výspi průběžných výsledků \\
        s, t & Zahájení nové serie testů
        p, 5 & Pozastavení běhu aplikace \\
        4 & Reakce na levou stranu \\
        6 & Reakce na pravou stranu \\
        c & Zahájení konfiguračního módu
    \end{tabular}
    \caption{Příkazy přijímané aplikací}
    \label{table:commands}
\end{table}






\section{Pinout}

Úloha počítá s přiřazením pinů mikrokontroleru uvedeným v tabulce \ref{table:pinout}. Je nezbyté připojit dvě tlačítka na uvedené piny proti zemi.
Diody je rovněž potřeba připojit externě přes ochraný rezistor (např 470 ohmů).

\begin{table}[htbp]
    \centering
    \begin{tabular}{c|c|c}
        signál & pin MCU & pin Nuclea \\ \hline
        pravé tlačítko & PA4 & A2 \\
        pause tlačítko & PA1 & A1 \\
        levé tlačítko & PA0 & A0 \\
        pravá LED & PC0 & A5 \\
        levá LED & PC1 & A4
    \end{tabular}
    \caption{Pinout aplikace}
    \label{table:pinout}
\end{table}

\section{Výsledky testu}
Výsledky testu jsou vypsány po ukončení testu, nebo na vyžádání pomocí klávesy \textbf{r}. Formát je patrný z následujícího příkladu:
\begin{lstlisting}
Test statistics: Duration 0 minutes and 41.915 seconds (paused 0.00 % of time).
 Left side hit  23 out of  28, accuracy 82.14 %. Reaction time:  271 ms average,  220 ms best.
Right side hit  17 out of  29, accuracy 58.62 %. Reaction time:  235 ms average,   35 ms best.
\end{lstlisting}

Pro každou stranu jsou samostatně zaznamenávány počty úspěsných a neúspěšných kol, průměrný a nejlepší reakční čas a přesnost.

\end{document}